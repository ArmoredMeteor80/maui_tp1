\documentclass[titlepage]{article}

\usepackage[utf8]{inputenc}
\usepackage[T1]{fontenc}
\usepackage[french]{babel}
\usepackage{hyperref}
\usepackage{lmodern}
\usepackage{xspace}
\usepackage{listings}
\usepackage{xcolor}
\usepackage{graphicx}
\usepackage[a4paper, left=2cm, right=2cm, top=2cm, bottom=2cm]{geometry}

%\usepackage{blindtext}
%\setcounter{tocdepth}{3}

\usepackage{mathtools}
\usepackage{amssymb}
%\usepackage{amsthm} %sert à faire des théorêmes  / corrolaires / lemmes etc...



\title{TP2: Sac à Dos; RSA}
\author{Mathieu Tuloup Mohamed Saidane}
\date{\today}

\begin{document}
\maketitle

\section{Sac à Dos: Chiffre de Merkle-Hellman}
\begin{enumerate}
    \item \begin{enumerate}
    \item Un sac à dos est dit facile si sa suite A est supercroissante c'est-à-dire que si pour n $\geq$ 1 on a : 
        $\sum_{k=0}^{n-1} a_k<a_n   $ en l'occurence la suite A est bien supercroissante
    
    \item Le modulo N est acceptable si la somme de la suite A est strictement inférieur à N:$\sum_{k=0}^{n-1} a_k < N=25646<25922$. Donc Le modulo est acceptable.
    \item Pour vérifier si un compliqueur est acceptable on va regarder si E est premier avec le module N. Si c'est le cas E peut être le compliqueur.
    \item Pour déterminer le Sac à dos difficile B, on a: $B_i=(A_i \times E)[N] $ avec E=10693 et N=25922 donc B=(9413,6596,11580,9500,15988)
    \item La faciliteur $D=E^{-1}[N]=20373 $ avec N=25922  
    \item La clé publique de Bob est B=(9413,6596,11580,9500,15988) le sac à dos difficile et N=25922
    \item Pour déchiffrer le message on va utiliser la clé privée, à savoir le sac à dos facile A ainsi que le faciliteur D.De plus on va utiliser l'algorithme glouton, à noter que A doit être supercroissante . On détermine C=$D\times 41577$. On applique l'algorithme glouton qui ici nous renvoie qu'il n'a pas de solution car $C\ne 0$
\end{enumerate}
 \item \begin{enumerate}
\item La suite A est bien supercroissante donc c'est un sac à dos facile.
 \item La somme de la suite A::$\sum_{k=0}^{n-1} a_k=103<N=105$. Le modulo N est acceptable.
 \item Bob peut prendre le compliqueur E=31 car 31 est premier avec le modulo N=105.
 \item B=(62,93,81,88,102,37)
\item $D=E^{-1}[N]=61$
\item La clé publique de Bob est B=(62,93,81,88,102,37) le sac à dos difficile et N=105
\item Pour chiffré un message de n=18 bits $M=(m_1,...,m_n)$, on calcul le cryptogramme. le message chiffré vaut 82.
 \end{enumerate} 

    
    
\end{enumerate}






\end{document}